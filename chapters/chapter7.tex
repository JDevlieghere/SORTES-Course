\documentclass[../main.tex]{subfiles} 

\begin{document}
\chapter{Implementing ASG}
In this chapter we look at how AS can be implemented on different architectures.

\section{Implementing ASG on a Naked Computer}
The proposed implementation is intended for rather simple systems running on a single processor. It can, of course be elaborated to suit more complex systems.
\subsection{Scheduling Parallel Components}
A naive way to schedule a set of tasks that may be appropriate in simple systems is the cooperative multitasking loop.
This is simply having a main loop run with the different tasks in it.
Depending on the relative frequency the task should run the task is called multiple times.
This kind of scheduling is non premptive meaning the OS will never interrupt a running task.
Because of this each task should be as short as possible.
The worse case respone time will be the sum of the maximum execution times of the functions.
A simple code example can be seen in \ref{ex:coopmulti} the comments there should explain what is being done.

%\includelst[ASGExample.c] << doesn't work inside lstinputlisting
\includelisting{ASGExample.c}{style=cstyle, caption=Example of cooperative multitasking where task A is run twice as frequent as task B. Each task would represent an ASG (sub)state. , label=ex:coopmulti}
%\lstinputlisting[style=cstyle, caption=Example of cooperative multitasking where task A is run twice as frequent as task B., label=ex:coop_multi]{../code/ASGExample.c}

\subsection{Implementing Rendez-vous}
Implementing rendez-vous is easy.
The state enumeration variables should be external (outside a function).
Assume that this is the condition to transition to state zzz in component t:
The state of component u is aaa or bbb while the state of comonent v is qqq.

Then the following can easily be checked in the function representing state t:
\begin{center}
\lstinline{if (((state\_u==aaa)||(state\_u==bbb))&&(state\_v==qqq)) state\_t=zzz;}
\end{center}

\subsection{Implementing Resource Management}
Can easily be done with external boolean variables: TRUE means available, FALSE means BUSY.
If there should only be transitioned to yyy if resource Y is available \lstinline{state=yyy} (from \ref{ex:coopmulti}) can be replaced with \ref{ex:res}.

\begin{lstlisting}[caption=Example of implementing resource management., label=ex:res]
	if(resource_R){
		resource_R=FALSE;
		state=yyy;
	} // Don't forget to FREE the resource in state yyyy!!!!
 }
\end{lstlisting}
\subsection{Handling MINVT \& Timeouts}
A software counter counting the amount of overflows in a hardware timer should be used.
The hardware timer usually generates an interrupt when it overflows. 
The most significant bits reside in the software and the least significant in the hardware timer.
Knowing the duration of a tick one can convert the duration of a minvt or timeout into ticks.

When entering a state with minvt or timeout the software time should be stored.
Then the current time can be checked in the xxx\_W state to see if the timeout is over or minvt has passed.


\section{Implementating ASG in C on MicroC/OS-II}

To implement an ASG diagram on top of MicroC/OS-II a couple of changes have to be made.
We'll discuss them on the basis of the OS function calls.
Instead of round robin scheduling the OS uses prioritized task scheduling. 
Because the OS uses static priorities only assign a higher priority to some task if what they do has a much shorter deadline than other tasks.
Since the scheduling is preemptive be careful with shared resources since at any moment any task can be suspended by a tasks with the higher priority. 

\begin{itemize}
	\item \lstinline{OSTaskCreate()} All tasks should be created in the main with this function.
	\item \lstinline{OSTaskSuspend()} Task can be temporairally suspended with this function. All tasks that do not need to run permanently should be put to sleep imediatly after being created.
	\item \lstinline{OSTaskResume()} Used to resume a task that has been suspended.
	\item \lstinline{OSTaskFlagPost()} Sets a certain flag to a certain value. Can be used to indicate that a resource is busy/free or a condition is satisfied.  
	\item \lstinline{OSTaskFlagPend()} Reads a certain flag and returns it value. This can be used to wait for a resource or to put the task to sleep till a trasition condition is satisfied.
	\item \lstinline{OSTaskTimeDLY()} Further execution of the task that calls this is temporairly suspended for the given time.
	\item \lstinline{OSTimeGet()} Returns the current time.
\end{itemize}

\subsection{Scheduling Parallel Components}
With the functions above this is an easy task to accomplish.
\subsection{Implementing Rendez-vous}
With \lstinline{OSTaskFlagPost()} conditions for the rendez-vous can be set and later read out with \lstinline{OSTaskFlagPend()}.
\subsection{Implementing Resource Management}
This still has to be implemented with mutexes or semaphores.

\subsection{Handling MINVT \& Timeouts}
If a state has a minvt, one must first get the value of the current time when entering the state (beginning of case xxx), using \lstinline{OSTimeGet()}, then, when when the action is finished (beginning of case xxx\_W one must again measure time with \lstinline{OSTimeGet()} and if the minvt is not over, one must wait for the remaining time with \lstinline{OSTimeDLY()}. 

Timeouts are not implemented by the OS. 
But it can easily be achieved in the application by creating a ``timeout'' task.
By default this task is suspended if another tasks that needs a timeout, the temporised task, it starts the timeout task (with \lstinline{OSTaskResume()}).
The timeout task itselfs then delays for the specified time. After that time it suspends himselfs and sets a flag to wake up the task it was waiting for. 
If the temporised task exists it's temporised state before the waiting is over it must notify the timeout task.
\end{document}

\documentclass[../main.tex]{subfiles} 

\renewcommand{\imageSrc}{../images/}

\begin{document}
\chapter{Design strategy for embedded real-time systems}

\section{Introduction}

\subsection{Time-related constraints}
The following constraints should be carefully examined before designing the system: 
\begin{itemize}
	\item Constraints related to the device under control of the system. 
	\item Constraints related to the computer(s) running the software. 
		\begin{description}
			\item[If you CAN'T choose this] first thing to check if the computer can meet the needed constraints of the device it will control.
				\item[If you CAN  chose this] Design software first assuming you have unlimited computing resources. Then select a suitable computing infrastructure.
		\end{description} 
	The goal of this strategy is to avoid that the infrastructure you selected is not able to meet the timing constraints. On top of that it helps to focus on the most important constraints of the device that will be controlled while designing the software.
\end{itemize}




\subsection{Software architecture}
Real-time software has a two-layer architecture: 
\begin{enumerate}
	\item real-time application layer
	\item real-time operating system layer
\end{enumerate}

Both layers are equally important in designing a real-time system and should be carefully designed or chosen.
The OS can be designed alongside the application or existing solutions can be used. 
Mostly using a specifically designed OS is generally faster since versatility costs time.

\section{Design Strategy}
The creation of a real-time system is done using 6 steps. These steps are listed below and each step is discussed further in this section. 
\begin{enumerate}
	\item Specify the problem. (see \ref{sss:specify_problem})
	\item Design the software architecture. (see \ref{sss:design})
	\item Identify the services the system layer must provide. (see \ref{sss:services})
	\item Validation or design of the system layer. (see \ref{sss:valid})
	\item Specify the data processing activities. (see \ref{sss:activities})
	\item Write the program(s). (see \ref{sss:write})
\end{enumerate}




\subsection{Specify the problem}
\label{sss:specify_problem}
The first and most important step in designing is specifying the problem. This is ``classical'' analysis work.
Errors mad in this step of the process are most costly so caution is advised. 
Try to do this without preselecting a solution.

The following models can help you in this step.
\begin{description}
	\item[structural model] Model of all the components, subsystems that can work in parallel.
	\item[behavioural model] Model of the desired behaviour e.g. state machines of components. 
\end{description}

\subsubsection{Analysis}
The analysis can cover an existing physical system (control systems are more frequently replaced than factories!) or specifications for a future physical system.

In the latter case the following should be considered.
\begin{enumerate}
	\item Are the specifications complete?
	\item Is the future system fully understood?
	\item Do the specifications cover exactly what the author meant?  
\end{enumerate}
Because of these questions the created model should be shown to the author of the specifications: He MUST understand and agree with the model.




\subsection{Design the software architecture}
\label{sss:design}
Next the software architecture can be designed by refining the behavioural model. 
Two types of systems can be distinguished:
\begin{description}
	\item[Reactive system] The desired behaviour of the controlled device depends on the commands from the system we are designing.
		In this case the behaviour of the software must be \emph{structurally similar} to that of the \textbf{controlled system}.
		The software sends commands to force the expected behaviour of the controlled device.
	\item[Data processing system] The behaviour of the software must be \emph{structurally similar} to that of the \textbf{monitored system} and is driven by events occurring in that system.
\end{description}

In both cases the following process should be followed:
\begin{itemize}
	\item Model structure of the physical system. 
	\item Make behavioural model of the physical system.
	\item Make behavioural model of the software system.
\end{itemize}

The last step should give you a rough skeleton of the application layer.

The tasks of the software should be identified.
A task can be seen as a sequence of states that may evolve parallel.
Almost always there are multiple possible sequences of states that represent the same global behaviour.
It is always possible to represent a behaviour with entirely sequential (mutually exclusive) set of states.
But this is not advised in general because the amount of states (and transitions) tends to get very high quickly.
That's why it's advised to try and \emph{maximalise parallelism} in the behavioural model.
This gives us a group of simpler sequential state machines evolving in parallel and interacting only when needed. The following interactions are possible:
\begin{itemize}
	\item information transfer
	\item synchronisations \item exclusions \item filiations (relationships?)
\end{itemize}


\subsection{Identify the services the system layer must provide}
\label{sss:services}
See what kind of (common) services should be required in the system layer. 
This is ``classical'' analysis work of the behavioural model used to deduce the needed system services




\subsection{Validation or design of the system layer}
\label{sss:valid}
In  this step the system layer is evaluated or created depending on whether an OS is available.

If an is OS available it should be evaluated as follows:
\begin{itemize}
	\item Does it provide the required services and respect the real-time constraints? 
	\item If not can it be adapted to satisfy these?
	\item If not can the proposed solution for the application layer be adapted to the possibilities of the available OS?
\end{itemize}

If no OS is available design the system layer so that it contains all the needed services of step 3.
Firstly the services that should be provided should be specified. 
Two categories of services exist: 
\begin{itemize}
	\item Basic services: To implement
	\begin{itemize}
		\item scheduler (can be very simple in some cases).
		\item low level routines: To act as an interface to the hardware (e.g. interrupt routines) or for basic interactions between tasks (e.g. synchronisation)
	\end{itemize}
	
	\item Derived services: Built on top of others (e.g. file systems).
\end{itemize}

Another categorisation is possible based on how services are activated: 
\begin{itemize}
	\item Activation by application layer. 
	\begin{itemize}
		\item With function calls: system layer must be linked directly to the application layer (single executable). 
		\item With system calls: Independent modules, Context switch is possible and both layers can reside in independent memory.  
	\end{itemize}
	\item External Events (e.g. When a character is received, activation by interrupts). 
\end{itemize}

\subsection{Specify the data processing activities}
\label{sss:activities}
Each of the behavioural states of the software should be complemented with descriptions of the data processing activities at that state.
How these activities are represented depends on the type of data to be processed.
Sometimes these descriptions can be as simple as a few lines of C/C++ or even a ladder diagram.




\subsection{Write the program(s)}
\label{sss:write}
The design was top down but the writing of the programs should be bottom-up. So first the system layer and next the application layer. 
Testing is vitally important so test as soon as as possible.

\section{Identification of suitable hardware}


\end{document}
